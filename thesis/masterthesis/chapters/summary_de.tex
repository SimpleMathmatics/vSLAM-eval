Im Kontext der automatisierten Erkundung und Modellierung einer 3D Umgebung, behandelt diese Arbeit zwei Forschungsfragen. Die meisten automatisierten Systeme unterteilen die automatisierte Erkundung in drei Teilprobleme: Lokalisierung, Kartierung, und Wegplanung \cite{accurat}. Die Lokalisierungs- und Kartierungsaufgabe wird mit den bereits existierenden SLAM Algorithmen abgedeckt. In diesem Zusammenhang wird in der ersten Forschungsfrage untersucht, welcher open-source SLAM Algorithmus am besten geeignet ist, um eine automatisierte Umfelds Erkundung mit Drohnen durchzuführen. Hierbei werden die von den Algorithmen  errechneten Flugkurven und Punktewolken sowie Rechenzeit untersucht. Die Ergebnisse haben gezeigt, dass ORB SLAM am besten geeignet ist für die oben beschriebene Aufgabe. Im zweiten Teil der Arbeit wurde analysiert, wie ein System implementiert werden kann, mithilfe dessen Nutzer ein automatisiertes Erkundungssystem in einer simulierten Umgebung entwickeln und testen können. In dem aus dieser Frage resultierende Framework können Nutzer eine Drohne, auf deren Kameradaten der ORB Algorithmus angewandt wird, innerhalb einer virtuellen Umgebung navigieren. Unter anderem der freie Zugriff auf die wahre Position der Drohne ermöglicht es Nutzern Wegplanungsalgorithmen auf die Drohne anzuwenden, um das automatisierte Erkundungssystem zu komplettieren.