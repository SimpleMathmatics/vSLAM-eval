\section{Discussion}

\subsection{Conclusion of SLAM-Algorithm Evaluation}

	The evaluation showed, that ORB SLAM outperforms DSO- and DSM SLAM regarding trajectory accuracy, point cloud accuracy and computation time. 
	Only when sequences are removed, where tracking was lost, DSM SLAM yielded a trajectory accuracy one cm better than ORB SLAM. The only major 
	drawback, when using ORB SLAM in the framework introduced in the next chapter, is the low number of generated points. However, these 
	points are more accurate and ORB SLAM reveals a lower computational time, which is also relevant for the later discussed path planning algorithm. 

	Computation is not yet in real time for ORB, however in order to lower the computation time, it is possible to decrease the frame 
	frequency, but this will also increase the risk of losing the tracking, since the steps between the frames are greater and 
	the constant velocity model cannot perform a sufficient accurate initial  prediction of the new position. Also, it is possible to change the hyper-parameters of 
	the algorithms to increase the computation speed. For example, instead of extracting 2000 features per image, 
	only 1000 can be considered. Finally, the computation speed can be increased by decreasing the resolution of the input images 
	or by upgrading the hardware with a better processor or a GPU. 

\subsection{Future Options}

In the recent years, machine learning solutions for the SLAM problem where examined \cite{cnnslam2} \cite{deepslam}. Thus, in the works "D3VO: Deep Depth, Deep Pose and Deep Uncertainty
for Monocular Visual Odometry" \cite{deepslam} and "CNN-SLAM: Real-time dense monocular SLAM with learned depth prediction" \cite{cnnslam2} Yang et al.
 and tateno et al. introduce the usage of neuronal nets to either support the depth, pose and and uncertainty estimation or even do the prediction only based on 
 the output of the Convolutional neuronal network. They claim to yield excellent results.

\begin{quote}
We systematically evaluated the VO performance of D3VO on the two datasets.
D3VO sets a new state-of-the-art on KITTI Odometry and
also achieves state-of-the-art performance on the challenging EuRoC MAV, rivaling with leading mono-inertial and
stereo-inertial methods while using only a single camera.
\end{quote}

Unfortunately, these works are closed-source and thus can not be used for the proposed framework in the second chapter. However, machine learning approaches 
hold great opportunities for the future, as they have been outperforming traditional approaches in most other computer vision related areas.  

\begin{quote}
Deep learning has swept most areas of computer vision
– not only high-level tasks like object classification, detection and segmentation \cite{p1} \cite{p2} \cite{p3}, but also low-level ones
such as optical flow estimation \cite{p4} \cite{p5} and interest point
detection and description \cite{p6} \cite{p7}. \cite{deepslam}
\end{quote}


