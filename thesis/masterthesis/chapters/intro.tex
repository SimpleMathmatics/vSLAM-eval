\chapter{Introduction}

Multiple applications exist for the autonomous exploration and mapping tasks using drones,
 such as search and rescue-, inspection- and surveillance operations \cite{usecases}. 
 
 The autonomous exploration task can be divided into three subproblems: localization, mapping, and path planning \cite{accurat}. 
 All three tasks should be performed simultaneously within an environment, which the drone has no information on a priori. 
 
 The localization task contains the estimation of the position of the drone within this environment and the mapping task refers to 
 the incremental creation of a 3-dimensional map. There are several methods that combine these two tasks in a so-called 
 simultaneous localization and mapping (SLAM) algorithm. The development of these SLAM algorithms is one of the most researched topics 
in the field of robotics \cite{slamintro}.

\begin{quote}
SLAM is used for many applications including mobile robotics, self-driving cars, unmanned
aerial vehicles, or autonomous underwater vehicles \cite{quote1}.
\end{quote}

 When combining a SLAM algorithm with a path planning algorithm, an autonomous exploration 
system is created. The autonomous exploration using SLAM and a path planning algorithm is sometimes also referred to as active SLAM \cite{inproceedings}. 
This active SLAM process is displayed in figure \ref{fig:autsys}. Details on how such a system can be initiated and terminated, can be found in section 
\ref{rosframe}.
 
 \fig{img/aut_system.png}{Automated exploration system, based on a SLAM algorithm and a path planning algorithm. }{fig:autsys}{0.9}
 
This work targets to answer two distinct research questions. 
 
 \begin{enumerate}
 \item{What is the most suitable open-source monocular visual SLAM algorithm for an exploration task?}
 
 
  This work is limited to evaluate monocular visual SLAM (vSLAM) algorithms, meaning that the algorithm is only working
 with a single RGB (red, green, blue) camera as sensor. Therefore, since nowadays RGB cameras are either a standard on drones or can easily be upgraded,
 these drones are very affordable, making it highly available for a larger user group. 
 
 
 Thus, in the first part of this work, three open-source monocular visual SLAM algorithms are evaluated. 
  DSO (Direct Sparse Odometry) SLAM, DSM (Direct Sparce Mapping) SLAM and ORB (Oriented FAST and Rotated BRIEF)
 SLAM were investigated regarding the predefined criteria of the accuracy of the resulting trajectory estimation, the point cloud accuracy and computational speed. 
 
 This was done by using the publicly available benchmark EuRoC dataset \cite{euroc}, containing video sequences filmed by a drone, the 
 ground truth of the position of the drone and the point cloud of the environment. Thus, the three SLAM algorithms are applied on the 
 video sequences, the resulting trajectories and map points are compared to the ground truth regarding the above-mentioned criteria.

 \item{What is a suitable framework to test and develop fully automated exploration systems within a simulated environment?}
 
 
 In the second part of this work, a Roboter Operating System (ROS) framework that enables users to develop a fully automated exploration system within 
 a virtual environment is proposed. This framework includes a process that provides a simulated Gazebo environment, making it possible to navigate 
 a virtual drone within a simulated environment. The sensors and behavior of the drone are modeled realistically. Most importantly, the drone is equipped with a RGB 
 camera, which allows the direct application of a vSLAM algorithm on the output. 
 Furthermore, the most suitable algorithm, evaluated in the first part of the work is implemented in a subprocess of this framework. 
 
 While the functionality and current state of the art of methods tackling the path planning task of the automated system are described and a suggestion on how it could 
 be implemented into the framework is given in section \ref{path}, this work does not include an actual implementation of such an algorithm. 
 Such implementations are left for further research. 
 
 The suggested framework should rather function as an option for users to implement and test out new path planning algorithm, providing optimal 
 prerequisites to do so. 
 
 For example, the subprocess of the framework, in which the path planning algorithm should run can be provided with all necessary data, 
 such as sensor data of the drone and the estimated orientation, position and point cloud by the vSLAM algorithm. This stream data is preprocessed 
 and standardized 
 in real time, enabling users to directly use it for their purposes. 
 
 
\end{enumerate} 
 
 




