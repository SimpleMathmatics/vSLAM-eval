\chapter{Introduction}

Multiple applications exist for the autonomous exploration and mapping tasks for drones;
 such as search and rescue-, inspection- and surveillance operations \cite{tasks}.
 The focus in this paper is on checking the feasabilty to perform an exploration and mapping task with a drone.
An approach with a combination of a vSLAM (visual simulatanious localization
and mapping) algorithm and a path planning algorithm is assessed. As the name 
suggests, the aim of a SLAM algorithm is to create a map of its completly
 unknown environment, while localizing itself within it only be using certain sensors.
 This paper is limited to monocular vSLAM, meaning that the algorithm is only working
 with a single RGB camera as sensor. This makes the drone very affordable, making it highly available for a larger user group.  
The main part of this work is the evaluation of vSLAM algorithms in order to find the 
most suitable method, that meets the requirements for this autonomous navigation task.
 DSO (Direct Sparse Odometry) SLAM, DSM (Direct Sparce Mapping) SLAM and ORB (Oriented FAST and Rotated BRIEF)
 SLAM were investigated regarding accuracy of the resulting trajectory estimation and pointcloud and computational speed. 
Furthermore, a ROS (Roboter Operating System) setup to combine the suitable vSLAM algorithm with a flight
 path planning algorithm within a simulated environment is introduced. Finally a recommendation
 for future work regarding 
the flight path algorithm is given.


\section{Related Work}

