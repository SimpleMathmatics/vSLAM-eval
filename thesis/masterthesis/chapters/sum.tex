\chapter{Summary}

In the context of automated exploration and mapping of a three dimensional environment with drones, in this work we answered two distinct research questions.
 
In order to examine what the most suitable open-source monocular vSLAM algorithm to be used for an automated exploration task is, ORB-, DSO- and DSM SLAM were evaluated regarding predefined criteria. The results of the evaluation, that were yielded after applying these algorithm on the benchmark EuRoC dataset, exhibit, that with respect to the trajectory accuracy, the point cloud accuracy and the computation time, ORB SLAM outperforms the other vSLAM algorithms. Although ORB SLAM generates significantly less points compared to DSO- and DSM-SLAM for the resulting point cloud, ORB SLAM reveals a computation speed at least three times as high as both other methods. ORB SLAM also shows better results regarding tracking accuracy and point cloud accuracy. 

Therefor, the proposed framework to answer the second research question, which targets to find a suitable framework to test and develop fully automated exploration systems within a simulated environment, relies on ORB SLAM to be a main component. The proposed framework is implemented in ROS and by using ORB SLAM, a gazebo simulation and further nodes, implemented specifically for this purpose, it is possible to develop and test flight path planning algorithms to complete the automated exploration and mapping task. In the current setup, a drone can be navigated in a simulated environment while the ORB SLAM algorithm is applied on the drones front camera output. By transforming the output of the ORB SLAM algorithm to the reference frame of the gazebo simulation, users of the framework are enabled to further develop an automated exploration system by implementing the fight path planning algorithm within the framework. 