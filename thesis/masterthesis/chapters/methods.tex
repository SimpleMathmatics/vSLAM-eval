\chapter{Methods}
\section{vSLAM Algorithms}

SLAM is one of the most emerging research topics in robotics \cite{slamintro}. It is applied in various applications, such as in augmented reality to estimate the camera pose,
autonomous navigation and computer vision-based online 3D modeling \cite{evolved} \cite{dso}.
The problem can be defined in a probabilistic way. The goal is to compute 

$$\mathbb{P}\left(m_{t+1},x_{t+1}|z_{1:t+1},u_{1:t}\right)$$

Where, as can be found in figure \ref{fig:slamover}, $m_{t+1}$ is the map (pointcloud of the surroundings) at timepoint $t+1$, $x_{t+1}$ the 
camera pose and position at timepoint $t+1$, $z_{1:t+1}$ all observations made to this timepoint and $u_{1:t}$ all historic control input. However, most
modern SLAM methods don't require the control input anymore. 

\fig{img/slamoverview.png}{Overview of the SLAM concept. Source: \cite{hist2}}{fig:slamover}{0.6}

With their work on the representation and estimation of spatial uncertainty \cite{hist1}
Smith et al created the first relevant work in the field on SLAM in 1986. However, due to lacking computational recources, the engagement in this topic stayed aminly
on a theoretical level. 


  \begin{quote}
   The result of this conversation was a recognition that
	consistent probabilistic mapping was a fundamental problem
	in robotics with major conceptual and computational issues
	that needed to be addressed. \cite{hist2}
  \end{quote}
  
This is because already then it was recognized, that camera pose and landmark positions had to be updated when moving further or optimizing the map and is a huge computational 
effort as the map grows. 

As time progressed, so did the computational resources and the algorithms became more advanced. Huge evolvements were achieved after 2010 \cite{evolved}, mainly because of the increased
use of augmented reality application, that may rely on real time vSLAM algorithms. The developed vSLAM methods can be devidet into two different classes: direct and feature based methods.

Direct algorithms use the entire image as input in order to compute the term in quotation \ref{eq:goal}. Therefore, the term is computed by optimizing the photometric error. 
This is the error, that results from comparing the intensities of each pixel after transformation (optimization) \cite{dso}. 

  \begin{quote}
	One of the main benefits of a direct formulation is that it
	does not require a point to be recognizable by itself, thereby
	allowing for a more finely grained geometry representation (pixelwise inverse depth). \cite{dso}
  \end{quote}

Feature based methods on the other hand, compute features for each frame, that serve as imput for further computation. These features usually are subsets of pixels, that have remarkable 
intensities and arrangement, such as corners (landmarks). These methods evaluate the upper term by computing the geometric error, since the feature positions are geomtric 
quantities. A main advantage of feature based methods is the rubostness over geometric distortions present in every-day-cameras \cite{dso}. 

Of all existing vSLAM algorithms, this work considers DSO, DSM and ORB SLAM for the evaluation of beeing a suited candide for flying a drone autonomously. This desicion is based on 
other research results in this area. % hier die gründe auflisten.

In the following section, an overview over how the algorithms work is given for each method. However, in order to understand this section, it is crutial to clarify basic definitions 
and vocabulary used in SLAM

	\subsection{Definitions}
		
		\subsubsection{Keyframe}
		
		Most SLAM Algorithms make usage of so called keyframes, as these keyframe-based approaches have proven to be more accurate \cite{keyframe}. 
	
		\subsubsection{Covisibility Graph}
		
		updating the edges resulting from the shared map
		points with other keyframes.
		
		\subsubsection{Group of Rigid Transformations in 3D}
		
		$\text{SE}(3)$ is the group of rigid transformations in 3D space \cite{se3}. Each Matrix $ T \in \mathbb{R}^{4x4}$ with 
		
		$$ T = \left[
		\begin{array}{rrr}
		R &  t \\  
		0  & 1 \\ 
		\end{array} \right]$$
		
		and $ R \in \mathbb{R}^{3x3}$ beiing a rotation matrix and $ t \in \mathbb{R}^{3}$ a translational vector, is an element of $\text{SE}(3)$.
		
		\subsubsection{Pointcloud}
		

	\subsection{ORB-SLAM}
	
	ORB SLAM is a feature based, state of the art slam method. The first version was published in 2015 \cite{orb}. 
	Here, an overview of the functionality of ORB SLAM is provided. The Algorithms runs on three threats simultanously.
	Each thread performes one of the following tasks: Tracking, Local Mapping and Loop Closing. An overview over the tasks can be found 
	in figure \ref{fig:orb_fig}. The explaination of these system components are described in the following subsections. 
	A more detailed explaination can be found in the paper of Raul Mur-Artal et al \cite{orb}.
	
	\fig{img/orb_slam_overview.png}{Overview of the system components extracted from \cite{orb}}{fig:orb_fig}{0.6}
	
	\subsubsection{Tracking}
	
	The tracking component determines the localization of the camera and decides, when a new keyframe is beeing inserted.
	As it is shown in figure \ref{fig:orb_fig}, the tracking is performed in four steps.
	
	\begin{enumerate}
	\item Feature Extracting 
	
	Features are extracted using Oriented FAST and Rotated BRIEF \cite{orb_feat}. This method starts by searching for 
	FAST (Features from Accelerated and Segments Test). Herefor, for each pixel x in the image, a circle of 16 pixels around that pixels
	is considered and checked if at least eight of these 16 pixels have major brightness differences. If so, the pixel x is considered as 
	a keypoint, since it is likely to be an edge or corner. This is repeated again and again after downsizing the image up to a scale of eight. 
	To extract features evenly distributed over the image, it is devided into a grid, trying to extract five features per cell. 
	Extracting features this way, makes the algorithm more stable to scale invariance. 
	Next the orientation of the extracted feature is calculated using a intensity centroid. 
	Finally the features are converted into a binary vectors (ORB descriptor) using a modified version, which is more robust to rotation, of BRIEF descriptors (Binary robust independent elementary feature).
	% wenn ich noch mehr brauch kann ich hier noch was adden. These ORB descriptors are then used for all feature matching tasks. 
	
	
	
	\item Initial Pose Estimation
	
	A constant velocity model is first run to predict to the camera pose. Then, the features of the last frame are searched. If no matches are found, 
	a wider area around the last position is searched. 
	
	\item Track Local Map 
	
	When the camera pose is estimated, map point correspondences are searched in 
	the local map, containing keyframes that contain the observed map points and
	the keyframes from the covisibility graph. The pose is then corrected with all
	matched mappoints. 
	
	% wenn ich noch mehr brauche, hier weitermachen.
	
	
	\item New Keyframe Decision 
	
	To insert the current frame as a keyframe, the following conditions have to be met:
	more than 20 frames have to be passed from the last relocalization or keyframe insertion (when not idle), 
    the current frame tracks at least 50 points or less than 90 percent of the points of the keyframe in the local 
    map with the most shard mappoints. 	
	
	
	\end{enumerate}
	

	
	\subsubsection{Local Mapping}
	
	Whenever a new Keyframe $K_i$ is inserted, the map is updated. 
	
	
	\begin{enumerate}
	\item{KeyFrame Insertion}
	
	The keyframe is inserted in the covisibility graph. Then the spanning tree
	is updated using using the keyframe with the most common points with $K_i$.
	Finally the keyframe is represented as a bag of words using the DBoW2 implementation. 
	Therefor, the image is saved by the number of occurances of features found in a a predefined
	vocabulary of features. When the vocabulary is created with images general enough, 
	it can be used for most environments.
	
	% cite

	
	\item{Recent Map Points Culling}
	
	A mappoint is removed from the map, when it is found in more than 
	25/% in the frames where it is predicted to be visible. Also, 
	it must be observed from more than two keyframes if more than one keyframe 
	has passed from map point creation. 
	
	\item{New Map Point Creation}
	
	A map point is created by calculating the triangulation of the connected
	keyframes in the covisibility graph. For each map point, the 3D coordinate 
	in the world coordinate system, its ORB descriptor, the viewing direction, 
	the maximim and minimum distance at which the point can be observed is stored. 
	
	\item{Local Bundle Adjustment}
	
	The keyframe poses $T_i \in \text{SE}(3)$ and Map Points $X_j \in \mathbb{R}^{3}$ 
	are optimized by minizing the reprojection error to the matched keypoints $x_{i,j} \in \mathbb{R}^{2}$.
	The error is computed by the following term:
	
	$$ e_{i,j} = x_{i,j} - \pi_i\left(T_i, X_j\right) $$. 
	$i$ is the respective Keyframe and j the index of the map Point. 
	$\pi_i$ is a projection function, calculation a transformation 
	to project all keypoints on mappoints by minimizing a cost function, that
	can be found in \cite{ba}.
	% citat anfang
	\begin{quote}
	In case of full BA
	(used in the map initialization) we optimize
	all points and keyframes, by the exception of the first
	keyframe which remain fixed as the origin. In local BA
	all points included in the local area
	are optimized, while a subset of keyframes is fixed. In
	pose optimization, or motion-only BA, all
	points are fixed and only the camera pose is optimized.
	\end{quote}
	At this point, a local BA is performed.
	
	\item{Local Keyframe Culling}
	
	With difference to other SLAM algorithm, ORB slam delets redundant 
	keyframes, which decreases computational efforts, since computational
	complexity grows with the number of keyframes. All keyframes are deleted, 
	where at least 90 percest of the map points can be found in at least three other 
	keyframes. 
	
	
	\end{enumerate}
	
	\subsubsection{Loop Closing}
	
	The loop closing is computed based on the last inserted keyframe $K_i$. 
	
	\begin{enumerate}
	\item{Loop Candidates Detection}
	
	First the similarity of $K_i$ to its neighbours in the covisibilty
	graph is computed by using the bag of words representation and a 
	loop candidate $K_l$ might be chosen. 
	
	\item{Similarity Transformation}
	
	In this step the transformation is computet, to map the map points
	from $K_i$ on $K_l$. Since scale can drift, also the scale is computet
	in addition to the rotation matrix and translation using the mothod of horn. 
	
	
	\item{Loop Fusion}
	
	Here, duplicated map points are fused and the keyframe pose $T_\omega$ is corrected by the transformation
	calculated in the previous step. All map points of $K_l$ are projected in $K_i$. 
	All keyframes affected by the fusion will update the edjes (shared map points) in the 
	covisibilty graph. 
	
	\item{ Essential Graph Optimization}
	
	Finally the loop closing error is distributed over the essential graph. 
	
	% essential graph hier noch beschreiben. 
	
	\end{enumerate}
	

	\subsection{DSM-SLAM}

	\subsection{DSO-SLAM}
	
	Direct Sparse Odometry was developed in 2016 by the Technische Universität München. % hier noch paar sätze dazu
	
	\subsubsection{Model Overview}
	
	\subsubsection{Frame Managemant}
	
	\subsubsection{Point Managemant}
 
 