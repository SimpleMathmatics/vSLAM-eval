\chapter{Methods}
\section{vSLAM Algorithms}

	\subsection{ORB-SLAM}

	\subsection{DSM-SLAM}

	\subsection{DSO-SLAM}

\section{Calculations}

	\subsection{Trajectory Alignment}
	
	In order to compare the evaluated position of the camera at a given time with the ground truth of the 
	position, the trajectories need to be aligned. This is because most SLAM Algorithms innitilize the origin
	of their coordinate system with the camera position from the first frame. Whereas the ground truth of the 
	trajectory uses a different origin. As a consequence, evaluated points $ \left\{{\widehat{p}_i}\right\}_{i=0}^{N-1} $ can not be 
	compared to the ground truth points $ \left\{{p_i}\right\}_{i=0}^{N-1} $
	Also, as described in the vSLAM Algorithms section,
	
	% input link of section here
	
	the minority of the existing vSLAM algorithms are recognizing the true scale of the coordinate system. For
	those two reasons, the target is to find $S = \left\{R,t,s\right\}$, while $R$ being a rotation matrix, $t$ a translation vector
	and $s$ a scaling factor, 
	such that
	
	$$ S = \argmin_{S' = \left\{R', t', s' \right\}} \sum_{i = 0}^{N-1} \norm{p_i - s'R'\widehat{p}_i - t'}^2 $$ .
	
	In other words, the evaluated points are rotated, translated and scaled in a way, that the sum squared error over the point
	distances is minimized. The upper expresssion is calculated by using the method of Umeyama \cite{ume}. 
	
	
	

\section{Datasets}

	\subsection{EuRoC Dataset}

	For the evaluation of the vSLAM Algorithms, the EuRoC dataset \cite{euroc} was used.
	The dataset contains eleven video sequences, recorded with a micro aerial vehicle at 20 frames per second. 
	For each Sequence, RGB images from two cameras exist. However, since the evaluation
	focuses on monocular SLAM methods, only the left camera was considered. Also the available 
	inertial and camera pose data was not taken in consideration. The first five sequences were recorded in 
	the machine hall at ETH Zürich, and the other six were recorded in a room, that was provided 
	with additional obsticals. For the latter six sequences, the groundtruth of the environment 
	exists as a dense pointcloud, as can be seen in figure \ref{fig:pcgt}.

	\fig{img/pointcloud_gt.png}{Pointcloud ground truth of sequence V1\_01\_easy visualized with python package pptk}{fig:pcgt}{0.7}

	Finally the true position of the 
	camera is known at a high frequency of over 200 points per second. 
	An overview of the sequences is shown in table \ref{table:euroctable}.



	\begin{table}
	\caption{Overview of the sequences included in the EuRoC Dataset}
	\begin{tabular}{ |p{3cm}||p{3cm}|p{3cm}|p{3cm}|  }
	\hline
	Sequence Name& Duration in $s$ & Average Velocity in $ms^{-1}$ &Pointcloud available\\
	\hline
	MH\_01\_easy & 182 & 0.44 & No\\
	MH\_02\_easy & 150 & 0.49 & No\\
	MH\_03\_medium & 132 & 0.99 & No\\
	MH\_04\_difficult & 99 & 0.93 & No\\
	MH\_05\_difficult & 111 & 0.88 & No\\
	V1\_01\_easy & 144 & 0.41 & Yes\\
	V1\_02\_medium & 83.5 & 0.91 & Yes\\
	V1\_03\_difficult & 105 & 0.75 & Yes\\
	V2\_01\_easy & 112 & 0.33 & Yes\\
	V2\_02\_medium & 115 & 0.72 & Yes\\
	V2\_03\_difficult & 115 & 0.75 & Yes\\
	\hline
	\end{tabular}
	\label{table:euroctable}
	\end{table}
	
\section{Setup and Environment}

	\subsection{Evaluation}
	
	The entire evaluation is run on a virtual machine. The host system is a lenovo yoga with eight GB of RAM and the basic model (8250U CPU @1.6 
	GHz 1.80GHz) of an eight core i5. The operating system of the host machine is Windows 10 Home. The virtual
	machine is given 5 GB of Ram and 4 cores. The operating system of the virtual machine is Ubuntu 18.04. All further setup information can be extracted 
	from the github repository.

	\subsection{Flight Path Planning}
 
 